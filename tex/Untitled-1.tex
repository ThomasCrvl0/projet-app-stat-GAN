\documentclass[letterpaper]{oup-contemporary}
% Packages divers
\usepackage{graphicx}
\usepackage{siunitx}
\usepackage[french]{babel}
\usepackage{bbm}
\usepackage{amsmath,amssymb}
\usepackage{stmaryrd}
% Les packages necessaires pour faire le pseudo code
%%%%%%%%%%%%%%%%%%%%%%%%%
\usepackage{algorithm}
\usepackage{algorithmic}

\renewcommand{\algorithmicrequire}{\textbf{Entrée(s) :}}
\renewcommand{\algorithmicreturn}{\textbf{retourner}}
\renewcommand{\algorithmicensure}{\textbf{Initialisation ;}}
\renewcommand{\algorithmicwhile}{\textbf{Tant que}}
\renewcommand{\algorithmicdo}{\textbf{Initialisation}}
\renewcommand{\algorithmicendwhile}{\textbf{Fin du Tant que ;}}
\renewcommand{\algorithmicend}{\textbf{Fin}}
\renewcommand{\algorithmicif}{\textbf{Si}}
\renewcommand{\algorithmicendif}{\textbf{Fin du si}}
\renewcommand{\algorithmicelse}{\textbf{Sinon}}
\renewcommand{\algorithmicelsif}{\textbf{Fin du sinon}}
\renewcommand{\algorithmicthen}{\textbf{Alors}}
\renewcommand{\algorithmicforall}{\textbf{Pour tout}}
\renewcommand{\algorithmicto}{\textbf{à}}
\renewcommand{\algorithmicendfor}{\textbf{Fin du pour}}
\renewcommand{\algorithmicdo}{\textbf{faire}}
\renewcommand{\algorithmicloop}{\textbf{boucler}}
\renewcommand{\algorithmicendloop}{\textbf{Fin de la boucle}}
\renewcommand{\algorithmicrepeat}{\textbf{Répéter}}
\renewcommand{\algorithmicuntil}{\textbf{Jusqu’à}}
\renewcommand{\algorithmicfor}{\textbf{Pour}}

\newcommand{\fonction}[5]{\begin{array}{l|rcl}
    #1: & #2 & \rightarrow & #3 \\
        & #4 & \mapsto & #5 \end{array}}
%%%%%%%%%%%%%%%%%%%%%%%%%%%%%%%%%%%%%%%%%%%%%%%%%%%%%%%%%%%%%%%
\setlength{\hoffset}{-18pt}        
\setlength{\oddsidemargin}{0pt} % Marge gauche sur pages impaires
\setlength{\evensidemargin}{9pt} % Marge gauche sur pages paires
\setlength{\marginparwidth}{54pt} % Largeur de note dans la marge
\setlength{\textwidth}{481pt} % Largeur de la zone de texte (17cm)
\setlength{\voffset}{-18pt} % Bon pour DOS
\setlength{\marginparsep}{7pt} % Séparation de la marge
\setlength{\topmargin}{0pt} % Pas de marge en haut
\setlength{\headheight}{13pt} % Haut de page
\setlength{\headsep}{10pt} % Entre le haut de page et le texte
\setlength{\footskip}{27pt} % Bas de page + séparation
\setlength{\textheight}{708pt} % Hauteur de la zone de texte (25cm)

\title{Generative Adversial Network}
\author[1]{Balsan Thibault, Carvaillo Thomas, L'archevêque Valentin}
\papercat{Projet d'Apprentissage Statistique}

\newtheorem{prop}{Proposition}

\begin{document}

\begin{frontmatter}
\maketitle
\begin{abstract}
Intro ici
\end{abstract}
\end{frontmatter}



\section{Partie ? - Modélisation mathématique}
Dans cette partie, nous allons apporter les divers éléments mathématiques de la méthode; en modélisant les perceptrons multicouches comme des distributions de probabilité, et en explicitant une méthode de construction optimale des perceptrons.\newline

Le perceptron multicouche générateur $G$ sera ici modélisé par la fonction différentiable  

\begin{center}
    $\fonction{G_{\theta_g}}{\mathbbm{R^n}}{E}{z}{x = G_{\theta_g}(z)}$
\end{center}

où $z \text{\textasciitilde} p(z)$ est le bruit donné en entrée du générateur et $E$ l'espace de l'échantillon d'apprentissage.\newline
Le discriminateur $D_{\theta_d}(x)$ sera représenté par la fonction toute aussi différentiable

\begin{center}
    $\fonction{D_{\theta_D}}{E}{\llbracket O,1 \rrbracket}{x}{\mathbbm{P}\left(x = G_{\theta_g}(z)\right)}$
\end{center}

Nous dénoterons par $p_{data}$ la distribution de probabilité de l'échantillon originel, et $p_g$ la distribution de probabilité de l'échantillon généré par $G$. 
Le but de ce GAN étant ainsi la convergence de $p_{data}$ vers $p_g$.
Ici, $D_{\theta_d}(x)$ ne retournera pas une valeur binaire, mais un scalaire compris entre $0$ et $1$, représentant la probabilité que $x$ soit ($D_{\theta_d}(x)=0$) ou non ($D_{\theta_d}(x)=1$) généré par le générateur. \newline
Avant de présenter la formalisation du problème, nous allons évoquer un cas élémentaire, en émettant l'hypothèse forte que la distribution $p_{data}$ est connue.
\begin{prop}[Optimalité de $D$]
    Soit $G_{\theta_g}$ un générateur \underline{fixe}, alors le discriminateur optimal $D^*_{\theta_d}(x)$ est définit par
    \begin{center}
        $D^*_{\theta_d}(x) = \frac{p_{data(x)}}{p_{data(x)} + p_g(x)}$ 
    \end{center}
\end{prop}
Une démonstration est présentée en annexe.\newline
Comme nous l'avons vu, il s'agit de maximiser la probabilité que $D$ ait raison, \underline{i.e.} maximiser $D_{\theta_d}(x)$, et, dans un même temps, que $G$ trompe $D$, \underline{i.e.} de minimiser $D_{\theta_d}(G_{\theta_g}(z))$.
Ceci peut se réécrire sous le problème d'optimisation suivant :
\begin{center}
    $\underset{G_{\theta_g}}{min}\left(\underset{D_{\theta_d}}{max}\left(\mathbbm{E}_{x \text{\textasciitilde} p_{data}}\left[log\left(D_{\theta_d}(x)\right)\right] + \mathbbm{E}_{z \text{\textasciitilde} p_z}\left[log\left(1 - D_{\theta_d}(G_{\theta_g}(z))\right)\right]\right)\right)$
\end{center}

Ce problème ce résout de manière computationnelle, à l'aide d'une descente de gradient pour le problème de minimisation et d'une ascension de gradient pour celui de maximisation. 

\begin{algorithm}
	\caption{\textbf{ - Goodfellow et al.}}
	\begin{algorithmic}[1]
        \REQUIRE{$T \in \mathbbm{N}$ le nombre d'itérations pour l'apprentissage, $K$ un hyperparamètre }
        \FOR{$t\in \{1,...,T\}$}
            \FOR{$k \in \{1,...,K\}$}
                \STATE{Générer aléatoirement un échantillon de bruit $z^{(1)}, ..., z^{(m)}$}
                \STATE{Extraire aléatoirement un échantillon $x^{(1)}, ..., x^{(m)}$ de l'échantillon originel}
                \STATE{Mettre à jour le discriminateur par ascension de gradient :
                    \begin{center}
                        $\Delta_{\theta_d}\displaystyle\frac{1}{m}\sum_{i = 1}^m \left( log\left[D_{\theta_d}(x^{(i)})\right] + log\left[1 - D_{\theta_d}(G_{\theta_g}(z^{(i)}))\right] \right)$
                    \end{center}
                } 
            \ENDFOR
            \STATE{Générer aléatoirement un échantillon de bruit $z^{(1)}, ..., z^{(m)}$}
            \STATE{Mettre à jour le discriminateur par ascension de gradient :
                \begin{center}
                    $\Delta_{\theta_g}\displaystyle\frac{1}{m}\sum_{i = 1}^m \left(log\left[1 - D_{\theta_d}(G_{\theta_g}(z^{(i)}))\right] \right)$
                \end{center}}
        \ENDFOR
	\end{algorithmic}
\end{algorithm}

%\bibliography{paper-refs}
\end{document}